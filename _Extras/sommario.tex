% !TeX spellcheck = en_US
\chapter*{Sommario} 
Grazie alle recenti innovazioni tecnologiche nel campo della microscopia miniaturizzata e, in particolare, allo sviluppo di una speciale tecnica che permette di misurare otticamente il livello intra-cellulare di ioni di calcio, si è resa possibile l'analisi dell'attività neuronale in risposta alla stimolazione esterna in animali svegli e liberi di muoversi.
Tuttavia, analizzare il livello di fluorescenza osservato presenta diverse complessità. Una prima difficoltà deriva dalla necessità di estrarre le serie del segnale (i cosiddetti \textit{spike train}), ovvero le serie di attività neuronale, pulite dall'errore di misura e dal lento decadimento del calcio. Dopodiché, il segnale estratto deve essere messo in relazione con con le condizioni sperimentali che l'hanno generato. Con questa tesi si vogliono introdurre degli approcci innovativi per l'analisi di dati di \textit{imaging} del calcio, nell'ambito di un'analisi statistica bayesiana.

L'approccio classico all'analisi di dati di \textit{imaging} del calcio si basa su una procedura in due passi: in una prima fase vengono estratti gli \textit{spike train}; successivamente, queste serie vengono messe in relazione alle condizioni esterne. Muovendoci all'interno di questo contesto, si introducono dei nuovi metodi computazionali per stimare in modo efficiente la relazione tra il segnale osservato e le condizioni sperimentali. In particolare, si pone l'interesse su modelli di regressione di Poisson, comunemente usati per studiare la dipendenza del numero di attivazioni da un insieme di fattori esterni. Si sviluppano un algoritmo Metropolis-Hastings e un \textit{importance sampler} per simulare dalla distribuzione a posteriori dei coefficienti di tali modelli, sotto l'assunzione di distribuzioni a priori Gaussiane sui parametri di regressione.

Un'analisi in due passi comporta alcuni svantaggi: per esempio, l'impossibilità di ottenere una chiara valutazione dell'incertezza complessiva, oltre all'impossibilità di condividere informazione tra le due fasi. Per questo motivo, si introduce un modello mistura che permette di stimare l'attività neuronale e, allo stesso tempo, di analizzare la distribuzione delle attivazioni in risposta a diverse condizioni sperimentali. In particolare, il modello proposto sfrutta una distribuzione a priori basata su due livelli annidati di misture finite, che permette di condividere l'informazione tra condizioni sperimentali, e indagare similitudini e differenze nella risposta ai diversi stimoli.

Infine, si introduce un'analisi multivariata di popolazioni di neuroni. In questo contesto l'interesse non è volto semplicemente ad estrarre e analizzare le singole serie di attività, ma anche ad indagare l'esistenza di gruppi di neuroni con modelli di attivazioni simili.
In questa tesi si introduce un'analisi multivariata che modella l'attività osservata come funzione di un processo latente continuo che descrive, per ogni istante temporale, la probabilità di osservare un'attivazione. Su questo processo continuo una distribuzione a priori mistura permette di identificare neuroni con uno schema di attivazioni simile per un periodo di diversi secondi.
