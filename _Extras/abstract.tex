\chapter*{Abstract} 
Recent advancements in miniaturized fluorescence microscopy have made it possible to investigate neuronal responses to external stimuli in awake behaving animals through the analysis of intra-cellular calcium signals. 
An ongoing challenge is deconvolving the noisy calcium signals to extract the spike trains, and understanding how this activity is affected by external stimuli and conditions.
In this thesis, we aim to provide novel approaches to tackle the various aspects of the analysis of calcium imaging data within the Bayesian framework.

Following the standard methodology to the analysis of calcium imaging data based on a two-stage approach, we investigate efficient computational methods to link the output of the deconvolved fluorescence traces with the experimental conditions. In particular, we focus on the use of Poisson regression models to relate the number of detected spikes with several covariates. We develop an efficient Metropolis-Hastings algorithm and importance sampler to simulate from the posterior distribution of the parameters of Poisson log-linear models under conditional Gaussian priors, with superior performance with respect to the state-of-the-art alternatives. 

Motivated by the lack of a clear uncertainty quantification resulting from the use of a two-stage approach, and the impossibility to borrow information between the two stages, we then develop a coherent mixture model that allows for the estimation of spiking activity and, simultaneously, reconstructing the distributions of the calcium transient spikes' amplitudes under different experimental conditions. More specifically, our modeling framework leverages two nested layers of random discrete mixture priors to borrow information between experiments and discover similarities in the distributional patterns of the neuronal response to different stimuli.

Finally, we move to the multivariate analysis of populations of neurons. Here the interest is not only to detect and analyze the spiking activity but also to investigate the existence of groups of co-activating neurons.
Estimation of such groups is a challenging problem due to the need to deconvolve the calcium traces and then cluster the resulting latent binary time series of activity. We describe a nonparametric mixture model that allows for simultaneous deconvolution and clustering of time series based on common patterns of activity. The model makes use of a latent continuous process for the spike probabilities to identify groups of co-activating cells. Neurons' dependence is taken into account by informing the mixture weights with their spatial location, following the common neuroscience assumption that neighboring neurons often activate together. 

% and to study their anatomical structure in the brain. To this end, we formulate a multivariate model where a mixture prior on a latent process on the spike probabilities allows us to cluster neurons on the basis of their pattern of activity over seconds-long periods of time. Moreover, following the common belief that neighboring neurons often show highly correlated activations, the mixture weights are informed using the spatial location of each neuron in the brain.