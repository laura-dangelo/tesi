\addchap{Introduction}
%\addcontentsline{toc}{chapter}{Introduction}
%\chapter*{Introduction} 


\addcontentsline{toc}{section}{Overview}
\section*{Overview}

A fundamental but unsolved problem in neuroscience is understanding the functioning of neurons and neuronal networks in processing sensory information, generating locomotion, and mediating learning and memory.
The investigation of the structure and function of the nervous system can be dated back to the nineteenth century with the invention of the technique of silver impregnation by Camillo Golgi in 1873, which allowed the visualization of individual neurons~\citep{drouin2015}. The technique initiated the study of the microscopic anatomy of the nervous system, and the investigation of how neurons organize to form the brain. 
Ever since there has been a significant research effort both to discover the cellular properties of the nervous system, and to characterize behaviors and correlate them with activity imaged in different regions of the brain.
However, many scientists recognize that despite the innovative techniques developed to observe and analyze neurons, we are still facing an ``explanatory gap'' between the understanding of elemental components and the outputs that they produce~\citep{parker2006,parker2010,dudai2004}. That is, we know a lot about the components of the nervous system, but still we have little insight into how these components work together to enable us to think, remember, or behave. One of the reasons of this gap is the availability of a huge quantity of data, but a lack of tools to integrate these data in order to obtain a coherent picture of the brain functioning~\citep{parker2010}.

The technological developments of the last few decades have opened fundamentally new opportunities to investigate the nervous system. Large neuronal networks can now be visualized using \textit{in vivo} high-resolution imaging techniques, which permit to record the neuronal activity in freely moving animals over long periods of time. In this thesis, we focus on data resulting from the application of the two-photon calcium imaging technique. Calcium ions generate intracellular signals that determine a large variety of functions in all neurons: when a neuron fires, calcium floods the cell and produces a transient spike in its concentration~\citep{grienberger2012}. By using genetically encoded calcium indicators, which are fluorescent molecules that react when binding to the calcium ions, it is possible to optically measure the level of calcium by analyzing the observed fluorescence trace. 
However, extracting these fluorescent calcium traces is just the first step towards the understanding of brain circuits: how to relate the observed pattern of neuronal activity to its output remains an open problem of research.

The set of tools that explicitly try to relate external stimuli with the brain activity are usually referred to as ``encoding models''. In this context the stimuli are considered as features, and they are used to predict patterns of neuronal activity. These models allow to investigate how experimental conditions and specific stimulation trigger the neurons' activity, and hence how external information is encoded by individual neurons and neuronal networks.



\addcontentsline{toc}{section}{Main contributions of the thesis}
\section*{Main contributions of the thesis}
The availability of large quantities of data from calcium imaging studies, and the relative scarcity of tools to analyze them, constitute a breeding ground for the investigation of new methodologies.
In this thesis, we aim to contribute to the development of novel statistical tools to gain new insights into the analysis of calcium imaging data.

\subsubsection{Efficient posterior sampling for Poisson regression}
Linear models and generalized linear models are among the most natural classes of encoding models~\citep{paninski2007}. They allow to link the observed output of an experiment with a number of features and experimental conditions in a flexible and interpretable way. In particular, if the variable of interest is the number of spikes, which is a proxy of the intensity of the neuronal response, Poisson regression represents a straightforward choice. However, the dimensionality of the considered data poses a computational challenge and leads to the need for efficient algorithms to obtain a sample from the posterior distribution of the parameters. 
Motivated by the lack of specific and efficient algorithms to sample from the posterior distribution of the parameters
of Bayesian log-linear models, in Chapter 2 we develop a novel sampling strategy which exhibits superior performance with respect to the state-of-the-art alternatives. 

In particular, we develop an efficient Metropolis-Hastings algorithm and importance sampler to simulate from the posterior distribution of the regression parameters.
The key for both algorithms is the introduction of a proposal density based on an approximation of the posterior distribution of parameters under conditional Gaussian priors. 
With conditional Gaussian prior, we refer to a possibly hierarchical prior with conditional distribution $\beta \sim \mathrm{N}(b, B)$, with $b$ and/or $B$ random. Examples include straightforward Gaussian prior distributions with informative $(b,B)$ fixed using prior information, and scale mixtures of Gaussian where $b$ is set to zero and the variance has a suitable hierarchical representation, such as the Bayesian lasso prior or the horseshoe prior, among others.

Our result leverages the negative binomial convergence to
the Poisson likelihood: thanks to this result, we are able to exploit the P\'olya-gamma data augmentation of~\citet{polson_scott_2013} to derive an efficient sampling scheme. 


