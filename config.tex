\usepackage{lipsum}


% Languages
\usepackage[english]{babel}
\usepackage{csquotes}

% Colors
\usepackage[usenames,dvipsnames,table]{xcolor}
\definecolor{UnipdRed}{RGB}{155,0,20}
\definecolor{Trasparentino}{gray}{0.4}

\usepackage{pgfplots, pdfpages}

% Fonts
\usepackage[no-math]{fontspec}
%\usepackage[OT1]{eulervm}
\usepackage{microtype,ifthen}
\usepackage{yfonts}

\defaultfontfeatures{%
    RawFeature={%
        +calt,   % *Contextual alternates
        +clig,  % *contextual ligatures
        +ccmp,  % *composition & decomposition
        +tlig,  % 'tex-ligatures': `` '' -- --- !` ?` << >>
        +cv06  % narrow guillemets
    }%
}
\setmainfont{TeX Gyre Pagella}
\newfontface\chapterNumber[Scale=2.5,Color=Trasparentino,Numbers=Lining]{TeX Gyre Pagella}
\newfontfamily{\smallcaps}[RawFeature={+c2sc,+scmp}]{TeX Gyre Pagella}
\newfontfamily{\swash}[RawFeature={+swsh}]{TeX Gyre Pagella}

\RequirePackage{textcase} % for \MakeTextUppercase
\makeatletter
\newcommand{\ct@altfont}{}
\newcommand{\ct@caps}{\ct@altfont\scshape}

\DeclareRobustCommand{\spacedallcaps}[1]{{\addfontfeature{LetterSpace=15.0}\ct@caps\MakeTextUppercase{#1}}} 
\DeclareRobustCommand{\spacedlowsmallcaps}[1]{{\addfontfeatures{LetterSpace=11.0}\ct@caps\MakeTextLowercase{#1}}} 
\makeatother

% Math stuff
\usepackage{amsmath,amssymb,amsfonts,amsthm,braket,cancel,bm}

\newtheoremstyle{definition}{}{}{\itshape}{\parindent}{\scshape}{}{1em}{\thmname{#1}\thmnumber{ #2}:}
\newtheoremstyle{theorem}{}{}{}{\parindent}{\scshape}{}{1em}{\thmname{#1}\thmnumber{ #2}:}
\theoremstyle{theorem}
\newtheorem{lemma}{Lemma}[section]
\newtheorem{corollary}[lemma]{Corollary}
\theoremstyle{definition}
\newtheorem{definition}{Definition}[section]

\DeclareMathOperator*{\argmin}{argmin}
\DeclareMathOperator*{\argmax}{argmax}
\DeclareMathOperator{\logm}{Log}
\DeclareMathOperator{\tr}{Tr}
\newcommand{\norm}[2][]{\left\Vert#2\right\Vert_{#1}}
\newcommand{\R}{\mathbb{R}}
\newcommand{\CC}{\mathcal{C}}
\newcommand{\Dir}{\mathrm{Dirichlet}}
\newcommand{\N}{\mathrm{N}}
\newcommand{\Mult}{\mathrm{Multinomial}}
\newcommand{\Beta}{\mathrm{Beta}}
\newcommand{\DP}{\mathrm{DP}}
\newcommand{\Unif}{\mathrm{Unif}}
\newcommand{\T}{ \mathrm{\scriptscriptstyle T} }
\newcommand{\ppg}{f_{\mbox{\textsc{pg}}}}
\newcommand{\Ca}{\mathrm{Ca}}

% Layout, graphics and tables
\usepackage{geometry}
    \geometry{%left=3.5cm, right=3.5cm, 
    	top=3.5cm, 
    	bottom=4cm, 
    	bindingoffset=0.5cm
    }
\usepackage{graphicx}
    \graphicspath{{_Images/}}
\usepackage{booktabs,caption,subcaption}
    \captionsetup{font=small,labelfont={sc},format=plain}

\newcommand{\gotoclearpage}{\cleardoublepage}

% Headers
\usepackage[automark]{scrlayer-scrpage}
\clearpairofpagestyles
\makeatletter
\let\MakeMarkcase\spacedlowsmallcaps
\renewcommand{\chaptermark}[1]{\markboth{\spacedlowsmallcaps{#1}}{\spacedlowsmallcaps{#1}}}
\renewcommand{\sectionmark}[1]{\markright{\textsc{\MakeTextLowercase{\thesection}}\hspace{1em} \spacedlowsmallcaps{#1}}}
\lehead{\mbox{\llap{\small\thepage\kern1em\color{Trasparentino}\vline}\color{Trasparentino}\hspace{0.5em}\headmark\hfil}}
\rohead{\mbox{\hfil{\color{Trasparentino}\headmark\hspace{0.5em}}\rlap{\small{\color{Trasparentino}\vline}\kern1em\thepage}}}
\rofoot[\mbox{\makebox[0pt][l]{\kern1em\thepage}}]{}
\renewcommand{\headfont}{\small}
\renewcommand{\pnumfont}{\small}
\clearscrplain
\def\toc@heading{\chapter*{\contentsname}
    \@mkboth{\spacedlowsmallcaps{\contentsname}}{\spacedlowsmallcaps{\contentsname}}}
\makeatother
\cfoot[\pagemark]{}

% Chapters, sections and stuff
\usepackage[clearempty,notocpart*]{titlesec}

\newcommand\formatchapter[1]{%
    \vbox to \ht\strutbox{%
        \setbox0=\hbox{\chapterNumber\thechapter\hspace{10pt}\vline\ }%
        \advance\hsize-\wd0 \advance\hsize-10pt\raggedright%
        \spacedallcaps{#1}\vss}}

\titleformat{\part}[block]{\normalfont\Huge}{Parte \thepart}{10pt}{\filcenter\scshape}
\titleformat{\chapter}[block]%
{\normalfont\Large}%
{{\chapterNumber\thechapter}%
    \ \,\hspace{10pt}\vline\ }{10pt}%
{\formatchapter}
\titleformat{\section}{\Large}{\scshape\thesection.}{1em}{\scshape}
\titleformat{\subsection}{\large}{\scshape\thesubsection.}{0.5em}{\bfseries} %\itshape
\titleformat{\subsubsection}{\large}{\quad\scshape\thesubsubsection}{0.5em}{\itshape}
\titleformat{\paragraph}[runin]{\itshape}{}{}{}

\titlespacing*{\section}{0pt}{22pt}{12pt}

%\renewcommand{\thesection}{\Roman{section}}
%\renewcommand{\thesubsection}{\thesection.\roman{subsection}}
%\renewcommand{\thesubsubsection}{\thesubsection.\roman{subsubsection}}
%\newcommand{\scref}[1]{\textsc{\ref{#1}}}

% Table of contents
\usepackage[titles]{tocloft}
%\setcounter{tocdepth}{1}
\renewcommand{\cftchapfont}{\scshape}
%\renewcommand{\cftchappagefont}{\bfseries}
%\renewcommand{\cftsecfont}{\bfseries}
%\renewcommand{\cftsecpagefont}{\normalfont}
%\renewcommand{\cftsubsecfont}{\normalfont}
%\renewcommand{\cftsubsecpagefont}{\normalfont}

%\renewcommand{\cftchapleader}{\hspace{1.5em}}%
%\renewcommand{\cftchapafterpnum}{\cftparfillskip}
%\renewcommand{\cftsecleader}{\hspace{1.5em}}%
%\renewcommand{\cftsecafterpnum}{\cftparfillskip}
%\renewcommand{\cftsubsecleader}{\hspace{1.5em}}%
%\renewcommand{\cftsubsecafterpnum}{\cftparfillskip}%

% Document data
\newcommand{\myTitle}{Bayesian modeling\\ of calcium imaging data}
\newcommand{\myShortTitle}{Thesis shortitle}
\newcommand{\mySubtitle}{Thesis subtitle}
\newcommand{\myName}{Laura D'Angelo}
\newcommand{\myDate}{25 Febbraio 2022}
\newcommand{\myProf}{Prof. Antonio Canale}
\newcommand{\myFaculty}{Corso di Dottorato di Ricerca in Scienze Statistiche}
\newcommand{\myDepartment}{Dipartimento di Scienze Statistiche}
\newcommand{\myUni}{Università degli Studi di Padova}
\newcommand{\myLocation}{Padova}


% Bibliography
\usepackage[style=authoryear,
			citestyle=authoryear,
			giveninits=true,
            backend=biber,
            backref=false,
            minbibnames=5, maxbibnames=5,
            mincitenames=1, maxcitenames=2,
            hyperref=true,
            uniquename=false, uniquelist=false,
            url=false, doi=false, isbn=false,
            date=year   
         ]
            {biblatex}
\DeclareFieldFormat{pages}{#1}

\renewbibmacro*{volume+number+eid}{%
	\printfield{volume}%
	%  \setunit*{\adddot}% DELETED
	%	\setunit*{\addnbspace}% NEW (optional); there's also \addnbthinspace
	\printfield{number}%
	\setunit{\addcomma\space}%
	\printfield{eid}}
\DeclareFieldFormat[article]{number}{\mkbibparens{#1}}

\renewbibmacro{in:}{}

\renewcommand*{\nameyeardelim}{\addcomma\space}
\DeclareNameAlias{author}{last-first}

\addbibresource{references.bib}



% Hyperref setup
\usepackage{hyperref}
\hypersetup{
%    pdftitle={\myTitle},    % title
%    pdfauthor={\myName},     % author
%    colorlinks=true,       % false: boxed links (set box color with linkbordercolor)
%    breaklinks,				% allow linebreaks on links
%    linkcolor=MidnightBlue,   % color of internal links
%    citecolor=TealBlue,       % color of links to bibliography
%    urlcolor=Sepia           % color of external links
    pdfauthor  = {name},
    pdfsubject = {PhD thesis},
    pdftitle   = {PhD thesis},
    pdfcreator = {name}
}
\renewcommand\UrlFont{\rmfamily}

\usepackage{breakurl}

\usepackage{algorithm, algpseudocode}


