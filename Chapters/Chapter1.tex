% Chapter 1
\chapter{Statistical modeling of calcium imaging data}

\fancyhead[RO,LE]{\thepage}
\fancyhead[LO]{Chapter 1 - \emph{Title of chapter}}
\fancyhead[RE]{Section \thesection \ - \emph{\Sectionname}}

\setlength{\parskip}{0.5pt}

\bigskip

\section{Overview of calcium imaging data} 

Calcium ions generate intracellular signals that control key functions in all types of neurons.
At rest, most neurons have an intracellular calcium concentration of about 100 nm; however, during electrical activity, the concentration can rise transiently up to levels around 1000 nm~\citep{berridge2000}. 
The development of techniques that enable the visualization and quantitative estimation of the intracellular calcium signals have thus greatly enhanced the investigation of neuronal functioning.
The development of calcium imaging techniques involved two parallel processes: the development of calcium indicators, which are fluorescent molecules that react when binding to the calcium ions, and the implementation of the appropriate imaging instrumentation, in particular, the introduction of two-photon microscopy~\citep{denk1990}.
In recent years, the innovations achieved in these two fields have allowed for real-time observation of biological processes at the single-cell level simultaneously for large groups of neurons~\citep{grienberger2012}. 

The output two-photon calcium imaging is a movie of time-varying fluorescence intensities, and a first complex pre-processing phase deals with the identification of the spatial location of each neuron in the optical field and source extraction~\citep{mukamel2009,dombeck2010}. The resulting processed data consist of a fluorescent calcium trace for each observable neuron in the targeted area which, however, is only a proxy of the underlying neuronal activity.
Hence further analyses are needed to deconvolve the fluorescence trace to extract the spike train (i.e. the series of recorded firing times), and to try to explain how these firing events are linked with the experiment that generated that particular pattern of activity.

There is currently a rich literature of methods addressing the issue of deconvolving the raw fluorescent trace to extract the spike train. A successful approach is to assume a biophysical model to relate the spiking activity, to the calcium dynamics, and to the observed fluorescence. \citet{vogelstein2010} proposed a simple but effective model that has later been adopted by several authors~\citep{pnevmatikakis2016, friedrich2016, friedrich2017, jewell2018, jewell2019}. The model considers the observed fluorescence as a linear (and noisy) function of the intracellular calcium concentration; the calcium dynamics is then modeled using an autoregressive process with jumps in correspondence of the neuron's firing events.
Denoting with $y_t$ the observed fluorescence trace of a neuron and with $c_t$ the underlying calcium concentration, for time $t=1,\dots,T$, the model can be written as
\begin{eqnarray}
&y_t = b + c_t + \epsilon_t,\quad \epsilon_t \sim \mathrm{N}(0,\sigma^2), \nonumber \\
&c_t = \gamma c_{t-1} + A_t + w_t, \quad w_t \sim \mathrm{N}(0, \tau^2),\nonumber
\label{eq:armodel}
\end{eqnarray}
where $b$ models the baseline level of the observed trace and $\epsilon_t$ is a Gaussian measurement error. In the absence of neuronal activity, the true calcium concentration $c_t$ is considered to be centered around zero. The parameter $A_t$ captures the neuronal activity: in the absence of a spike ($A_t = 0$), the calcium level follows a AR(1) process controlled by the parameter $\gamma$; when a spike occurs, the concentration increases instantaneously with the spike amplitude $A_t > 0$.


\subsection{Allen Brain Observatory data}
\noindent

\section{A brief review of some Bayesian nonparametric models} 
\noindent

\noindent

