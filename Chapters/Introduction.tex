% Introduction
\addcontentsline{toc}{chapter}{Introduction}
\chapter*{Introduction} % Write in your own chapter title



\fancyhead[RO,LE]{\thepage}
\fancyhead[LO]{\emph{Introduction}}
\fancyhead[RE]{\emph{Overview}}

\setlength{\parskip}{0.5pt}

\bigskip

\addcontentsline{toc}{section}{Overview}
\section*{Overview}

A fundamental but unsolved problem in neuroscience is understanding the functioning of neurons and neuronal networks in processing sensory information, generating locomotion, and mediating learning and memory.
The investigation of the structure and function of the nervous system can be dated back to the nineteenth century with the invention of the technique of silver impregnation by Camillo Golgi in 1873, which allowed the visualization of individual neurons~\citep{drouin2015}. The technique initiated the study of the microscopic anatomy of the nervous system, and the investigation of how neurons organize to form the brain. 
Ever since there has been a significant research effort both to discover the cellular properties of the nervous system, and to characterize behaviors and correlate them with activity imaged in different regions of the brain.
However, many scientists recognize that despite the innovative techniques developed to observe and analyze neurons, we are still facing an ``explanatory gap'' between the understanding of elemental components and the outputs that they produce~\citep{parker2006,parker2010,dudai2004,paninski2018}. That is, we know a lot about the components of the nervous system, but still we have little insight into how these components work together to enable us to think, remember, or behave. One of the reasons of this gap is the availability of a huge quantity of data, but a lack of tools to integrate these data in order to obtain a coherent picture of the brain functioning~\citep{parker2010}.

The technological developments of the last few decades have opened fundamentally new opportunities to investigate the nervous system. Large neuronal networks can now be visualized using \textit{in vivo} high-resolution imaging techniques, which permit to record the neuronal activity in freely moving animals over long periods of time. In this thesis, we focus on data resulting from the application of the two-photon calcium imaging technique. Calcium ions generate intracellular signals that determine a large variety of functions in all neurons: when a neuron fires, calcium floods the cell and produces a transient spike in its concentration~\citep{grienberger2012}. By using genetically encoded calcium indicators, which are fluorescent molecules that react when binding to the calcium ions, it is possible to optically measure the level of calcium by analyzing the observed fluorescence trace. 
However, extracting these fluorescent calcium traces is just the first step towards the understanding of brain circuits: how to relate the observed pattern of neuronal activity with its output remains an open problem of research.



\phantomsection
\addcontentsline{toc}{section}{Main contributions of the thesis}
\fancyhead[RE]{\emph{Main contributions of the thesis}}
\section*{Main contributions of the thesis}


\noindent

